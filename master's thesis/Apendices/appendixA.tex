\chapter{Definiciones y Acronimos}
\label{Appendix:1}

\subsection{Definiciones}

El gobierno de España define los \textbf{datos de alto valor} como “documentos cuya reutilización está asociada a considerables beneficios para la sociedad, el medio ambiente y la economía, en particular debido a su idoneidad para la creación de servicios de valor añadido, aplicaciones y puestos de trabajo nuevos, dignos y de calidad, y al número de beneficiarios potenciales de los servicios de valor añadido y aplicaciones basados en tales conjuntos de datos” Esta definición nos ofrece varias pistas sobre la manera en la que se prevé que se identifiquen esos conjuntos de datos de alto valor a través de una serie de indicadores que incluirían:
\begin{itemize}
	\item Su potencial para generar beneficios sociales o medioambientales significativos.
	
	\item Su potencial para generar beneficios económicos y nuevos ingresos.
	
	\item Su potencial para generar servicios innovadores.
	
	\item Su potencial en cuanto a número de usuarios beneficiados, con atención particular a las PYMEs.
	
	\item Su capacidad para ser combinados con otros conjuntos de datos
\end{itemize}

\subsection{Acronimos}


 \newpage
