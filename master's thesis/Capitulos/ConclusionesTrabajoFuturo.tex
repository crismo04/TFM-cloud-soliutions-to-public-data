\chapter{Conclusiones y Trabajo Futuro}
\label{cap:conclusiones}


[TODO, importante a tener en cuenta:
Señalar los principios y relaciones que indican los resultados (qué es lo que se ha sacado en claro con la investigación, futuras implicaciones que se pueden extraer, etc.).
· Relacionar los resultados con otros trabajos publicados.
· Hay que mencionar también las excepciones, faltas de correlación o aspectos no resueltos.
· Indicar futuras líneas de trabajo
]

\section*{Trabajo futuro}

Añadir al estudio un coste ecológico de las tecnologías.

Contactar con responsables de algunos de los estudios citados y conducir una encuesta a los usuarios para comprender la utilidad de los datos.

Contactar con las nubes privadas europeas y españolas, ya que aunque es normal que en su mayoría no ofrezcan planes gratuitos para evitar su abuso, es posible que contactando como entidad investigadora dieran acceso a las cloud para poder comprobar su utilidad en este estudio. También con las iniciativas como BDTI de la Unión Europea \citep{BDTIEuropeProject} para probar proyectos como este.

Construir interfaz de usuario para visualizar los datos de una manera mas optima, ya sea con Firebase, o una instancia de Superset.