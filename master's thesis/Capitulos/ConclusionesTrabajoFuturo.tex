\chapter{Conclusiones y Trabajo Futuro}
\label{cap:conclusiones}

El objetivo principal de este proyecto era, como ya hemos dicho, desarrollar una aplicación que recopile datos, dándoles un sentido extra, para después mostrarlos de una forma que resulte útil para la creación de un libro de vida. Después de implementar toda la aplicación, podemos afirmar que hemos conseguido cumplir en gran medida los objetivos con nuestra solución, ya que finalmente hemos analizado qué redes sociales, y que datos de estas, son los más convenientes para la creación de un libro de vida, descargárnoslos y tratarlos, dándoles aún más sentido y valor, mostrándolos después en una interfaz con el formato de un libro de vida. Esperamos haber contribuido a la sociedad en algo positivo, ayudando a personas que de verdad lo necesitan y facilitando la vida de profesionales como lo son los terapeutas ocupaciones. 

En la parte técnica, creemos que los datos que hemos obtenido tienen un valor añadido sobre los trabajos ya existentes en este campo y que cosas como la puntuación de los usuarios o páginas puede ser de mucha utilidad. Como contraparte, sabemos que es prácticamente inevitable perder ciertas funcionalidades pensadas en la idea inicial, como el tratamiento de más datos demasiado grandes para el alcance de este proyecto, aunque hemos intentado mitigar este impacto de la mejor forma posible.

Por otro lado, podemos afirmar sin ninguna duda que este trabajo nos ha servido para crecer como desarrolladores, ya que hemos trabajado partiendo de cero, sin ningún tipo de guión para la realización del proyecto y teniendo que buscar la información y documentación necesaria por nuestra cuenta. Mencionar por supuesto lo orgullosos que estamos de todo lo aprendido con las distintas tecnologías, lenguajes y herramientas empleadas, las cuales estamos convencidos de que volveremos a utilizar en el futuro en nuestros proyectos personales y laborales. También los conocimientos aprendidos en ámbitos como el webScrapping, los libros de vida y los problemas de perdida de memoria, adquiriendo una sensibilización especial con estos problemas tan actuales.

\section{Trabajo futuro}

Durante el trabajo que hemos realizado, hemos dejado funcionalidades o mejoras en el camino, que se podrían realizar en el futuro para mejorar el desempeño de la aplicación, así como cosas a las que es posible que haya que realizarles un mantenimiento periódico:

\subsection*{Descarga de datos}
En este apartado, respecto a Facebook, hay poco que creamos que se pueda hacer. Quizás un maestro del Scraping web consiga automatizar todo el trabajo que la empresa ya hace recopilando los datos de sus usuarios. Puede que la mejor forma de ampliarlo respecto a esta red social sea convencer a los familiares y amigos más cercanos para que sumen sus datos a los que ya disponemos y crear así una base de datos mucho más potente.

También se podría considerar la extracción de datos de otras fuentes de datos, como puede ser APIs o trabajos ya existentes en Twitter o los datos que otras aplicaciones diferentes a Facebook y Google recopilan del usuario, que también pueden tener algún valor para la creación de libros de vida, así como los datos de estas que no hemos considerado oportuno tratar, pero pueden aportar valor.

\subsection*{Creación de las tablas, parseo de JSON e inserción}
 Esta sección se ampliaría a costa de las anteriores, es decir dependiendo del tamaño de las nuevas fuentes de datos que se sumen a la aplicación.
 
\subsection*{Tratamiento de los datos}
Con este apartado estamos contentos en el resultado, aun así, se podría mejorar la técnica de análisis de sentimiento, a costa de tiempo de ejecución o el aprovechamiento de investigación futura en este campo para el idioma español. Como ya hemos dicho en el capítulo 2, esto podría ser algo muy interesante que mirar en el futuro, ya que hay trabajos que ya están consiguiendo analizar los sentimientos de textos en español de forma muy eficaz, y que es posible que añadir esa potencia de clasificación de emociones al proyecto que nosotros hemos llevado a cabo, incremente en gran medida la puntuación que les damos a elementos como los amigos, los lugares, Etc. además de dar valor a cosas que hemos descartado como los mensajes privados. También, si tuviéramos en cuenta la ampliación en cuanto a la descarga de datos, este apartado también debería crecer en tamaño con respecto a esos nuevos datos descargados, para enlazarlos de nuevas maneras y aportar más valor.

\subsection*{Interfaz de usuario}
En la primera versión con Tkinter, la interfaz cumplía su funcionalidad, pero no era lo más vistoso para un usuario, una vez pasado a un formato web, se ha visto una mejora sustancial para la visualización de los datos, pero dada la amplitud de la comunidad de React, es seguro que si se pudiera contar en el equipo con un diseñador y con un experto en React, la aplicación podría llegar a unos niveles superiores. También, aunque la aplicación se ve correctamente en dispositivos móviles, la necesidad de Flask y Python para obtener los datos hace que no se puedan mostrar en estos dispositivos, por lo que una mejora seria añadir toda la información a un servidor, para que se leyeran los datos desde ahí y poder visualizar la aplicación en su formato móvil.

\subsection*{Usabilidad de la aplicación}
Somos conscientes de que utilizamos varias tecnologías entrelazadas y que es muy posible que un usuario que no esté familiarizado con la tecnología no pueda instalarse todos los componentes necesarios para su utilización, por lo que se podrían realizar varias mejoras cambiando las tecnologías utilizadas con el fin de unificarlas o aunando todas las que ahora utilizamos. Esta última parte podría realizarse subiendo la aplicación a un servicio de hosting y simplemente habría que hablar con el hosting para que nos instalase las tecnologías y librerías citadas anteriormente. También podría Dockerizarse toda la aplicación para que el usuario simplemente tuviera que instalarse Docker y hacer un Docker push y Docker run para poder utilizarla.