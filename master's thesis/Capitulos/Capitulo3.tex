\chapter{Materiales y métodos}
\label{cap:Materiales y metodos}

[TODO, importante a tener en cuenta:
 -  detallarse cada paso que se ha dado para llegar a los resultados describiendo, en orden lógico y expresado con claridad, los materiales y recursos empleados. 
 - No avanzar resultados y redactarse en pasado

]

En este capítulo vamos a describir el proceso que se ha seguido en la realización del trabajo, las distintas tecnologías, lenguajes de programación y herramientas, así como las que se ha valorado pero descartado. También se definirán los métodos de desarrollo, aplicaciones e incluso modelo de trabajo.



\section{Materiales}

[TODO, herramientas, programas y material utilizado, incluyendo por ejemplo los tipos de IA]

\subsection{Lenguajes} 

\subsubsection*{PYTHON}
Python es un lenguaje de programación interpretado y centrado en la legibilidad de su código. Se trata de un lenguaje de programación multiparadigma, ya que soporta parcialmente la orientación a objetos, programación imperativa y, en menor medida, programación funcional.

\subsubsection*{SQL}
SQL es un lenguaje de dominio específico utilizado en programación, diseñado para administrar, y recuperar información de sistemas de gestión de bases de datos relacionales. Es un sistema que facilita el tratamiento de datos, así como la separación de estos datos del programa principal, permitiendo tener más modularidad.
Utilizamos SQL para almacenar información, así como para extraer esta misma información, tratarla y almacenarla ya tratada en la base de datos. 

\subsubsection*{\LaTeX}
\LaTeX\space es un sistema de composición tipográfica de alta calidad que incluye funcionalidades diseñadas para la producción de documentación técnica y científica. Es el estándar de facto para la comunicación y publicación de documentos científicos, el cual nos ha permitido desarrollar una memoria profesional y facilitar el diseño sin tener que preocuparnos por la forma cada vez que añadíamos cambios.       
Hemos usado \LaTeX\space para desarrollar este documento en la aplicación de TeXstudio y el compilador MikteX.

\subsubsection{Lenguajes descartados} 

[TODO]

\subsection{Herramientas} 

\subsubsection*{Visual Studio Code}
Visual Studio Code es un editor de código fuente desarrollado por Microsoft para Windows, Linux y MacOS. Incluye soporte para la depuración, control integrado de Git, resaltado de sintaxis, finalización inteligente de código, fragmentos y refactorización de código. 

Utilizamos Visual Studio Code como entorno de desarrollo software, ya que comparándolo con otras alternativas que nos brindan en la carrera, lo creemos bastante más útil, sobre todo para la programación en Python o React.

Lo que nos ha hecho decantarnos por él por encima del resto, es la gran comunidad que tiene detrás, la cual cuenta con un gran número de tutoriales y extensiones que nos facilitan mucho la programación y la integración con otras aplicaciones como Github. También destacar su intérprete, para probar pequeños fragmentos de código, lo cual nos ha ahorrado tiempo en depuración de errores.

\subsubsection*{Github}
GitHub es una forja para alojar proyectos utilizando el sistema de control de versiones Git. Se utiliza principalmente para la creación de código fuente de programas de ordenador.  

Utilizamos GitHub como sistema de control de versiones y repositorio de código por su tremenda utilidad para comunicarnos y trabajar en paralelo, lo cual ha sido una necesidad en los tiempos de pandemia en los que este proyecto se ha realizado. Esto nos ha asegurado no perder nada de progreso y llevar un control del avance del proyecto en todo momento. Además nuestros tutores nos facilitaron un repositorio privado en el que nos asegurábamos la seguridad del código, por lo que su uso era casi una obligación frente a otras alternativas.

\subsubsection*{TeXstudio y MiKTeX}
TeXstudio es un editor de \LaTeX\space de código abierto y Multiplataforma con una interfaz similar a Texmaker. TeXstudio es un IDE de \LaTeX\space que proporciona un soporte moderno de escritura, como la corrección ortográfica interactiva, plegado de código y resaltado de sintaxis, por lo que lo hemos considerado ideal para la elaboración de este documento.
Mientras que MiKTeX es el gestor de paquetes integrado, que instala los paquetes que hacen falta para el correcto funcionamiento de TeXstudio y para la creación de este documento.

\subsection{Herramientas descartadas} 

[TODO]

\section{Metodos}

Para llegar a nuestro objetivo de diseño, hemos dividido la implementación en diferentes módulos:

\begin{itemize}
	\item \textbf{ búsqueda y almacenamiento de datos.}
	\item \textbf{ Tratamiento básico de los datos.}
	\item \textbf{ Estudio con modelos de IA en diferentes nubes}
	\item \textbf{ Comparación y estudio de resultados.}
\end{itemize}

\subsection{Utilización de la solución}
