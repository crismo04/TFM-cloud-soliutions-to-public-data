\chapter{Conclusions and Future Work}
\label{cap:conclusions}

The main objective of this project was, as we have already said, to develop an application that collects data, giving them an extra meaning, and then display them in a way that is useful for the creation of a life book. After implementing the whole application, we can say that we have largely achieved our objectives with our solution, as we have finally analysed which social networks, and which data from these networks are the most suitable for the creation of a book of life, downloading and processing them, giving even more meaning and value to these data, displaying them afterwards in an interface with the format of a book of life. We hope to have contributed to society in a positive way, helping people who really need it and making life easier for professionals such as occupational therapists.

On the technical side, we believe that the data we have obtained has an additional value over existing work in this field and that things like user or page ratings can be of great use. On the other hand, we know that it's practically inevitable to lose certain functionalities thought in the initial idea, such as the processing of more data, which is too big for the scope of this project, although we have tried to mitigate this impact in the best possible way.

On the other hand, we can say without any doubt that this work has helped us to grow as developers, since we have worked starting from zero, without any kind of guide for the realization of the work and having to look for the necessary information and documentation on our own. Of course, we must mention how proud we are of everything we have learned with the different technologies, languages and tools used, which we are convinced that we will use again in the future in our personal and work projects. Also the knowledge learned in areas such as webScrapping, life books and memory loss problems, acquiring a special awareness of such current issues.

\section{Future work}

During the work we have done, we have left functionalities or enhancements along the way, which could be done in the future to improve the performance of the application, as well as things that may need to be maintained on a regular maintenance basis:

\subsection*{Downloading data}
In this section, with respect to Facebook, there is little that we think can be done. Perhaps a web scraping master will be able to automate all the work the company already does collecting data from its users. Perhaps the best way to scale it up with respect to this social network is to convince family and close friends to add their data to what we already have and create a much more powerful database.

It could also be considered the extraction of data from other data sources, such as apis or jobs already existing in Twitter, or data that other applications other than Facebook and Google collect from the user, which may also have some value for the creation of life books, as well as data from these that we have not considered appropriate to treat, but can provide value.

\subsection*{Tables creation, JSON parsing and insertion}
This section would be expanded at the expense of the previous ones, that is, depending on the size of the new data sources added to the application.
 
\subsection*{Data processing}
With this section we are happy in the result, even so, the sentiment analysis technique could be improved, at the cost of execution time or the exploitation of future research in this field for the Spanish language. As we have already said in chapter 2, this could be something very interesting to look at in the future, as there are works that are already managing to analyze the sentiments of Spanish texts very effectively, and that it is possible that adding that power of emotion classification to the project that we have carried out, will greatly increase the score we give to elements such as friends, places, Etc. as well as giving value to things that we have discarded like private messages. Also, if we were to take into account the expansion in terms of data download, this section should also grow in size with respect to that new data, to link them in new ways and provide more value.

\subsection*{User interface}
In the first version with Tkinter, the interface was functional, but it was not the most attractive for a user, once moved to a web format, we have seen a substantial improvement for the visualization of the data, but given the breadth of the React community, it is certain that if you could count on the team with a designer and a React expert, the application could reach higher levels. In addition, although the application looks good on mobile devices, the necessity of Flask and Python to obtain the data means that it cannot be displayed on these devices, so an improvement would be to add all the information to a server, so that the data can be read from there and the application can be displayed in its mobile format.

\subsection*{Application usability}
We are aware that we use several intertwined technologies and that it is quite possible that a user who is not familiar with technology may not be able to install all the necessary components for its use, so various improvements could be made by changing the technologies used in order to unify them or by bringing together all the ones we now use. This last part could be done by uploading the application to a hosting service and we would simply have to talk to the hosting to install the technologies and libraries mentioned above. The whole application could also be Dockerized so that the user would simply have to install Docker and do a Docker push and Docker run to be able to use it.

