\chapter{Introducción}
\label{cap:introduccion}

\chapterquote{We can only see a short distance ahead, but we can see plenty there that needs to be done.}{Alan Turing}

\section{Motivación}

Empezaremos por el principio, definiendo que son los tres principales elementos de este proyecto [TODO]


\section{Objetivos y alcance}

El alcance de este proyecto es, por un lado [TODO]

\section{Plan de trabajo}


Una vez definido el alcance, me gustaría destacar las cinco fases en las que se ha dividido el proyecto, que se han ido iterando para la creación de varios prototipos funcionales:


\begin{enumerate}
	
	\item \textbf{Fase de investigación académica:} Búsqueda de estudios o trabajos a cerca del estado actual de los datos abiertos, las tecnologías en la nube y modelos o herramientas de inteligencia artificial.
	
	\item \textbf{Fase de investigación técnica:} Búsqueda de información a cerca de diferentes fuentes publicas de datos,  tecnologías en la nube y modelos o herramientas de IA que nos ayuden a tratar, filtrar y entender todos los datos públicos recopilados.
	
	\item \textbf{Fase de análisis de requisitos:} [TODO]
	
	\item \textbf{Fase de implementación:} [TODO]
	
	\item \textbf{Fase de pruebas:} [TODO]
	
	\item \textbf{Memoria:} Elaboración de este documento, plasmando las fases anteriores en texto y especificando el desarrollo del proyecto y los resultados del mismo.
\end{enumerate}

\section{Estructura de esta memoria}

Toda esta memoria se ha construido con \LaTeX\space [\ref{latexDef}] y ayuda de la plantilla \texis,\space El resto de la memoria se estructurara por capítulos de esta manera:

Capitulo \ref{cap:estadoDeLaCuestion} Estado de la cuestión, donde se plasmaran las conclusiones de las primeras dos fases de investigación.

Capitulo \ref{cap:Materiales y metodos} Materiales y métodos [TODO]

Capitulo \ref{cap:Resultados} Resultados [TODO]

En próximos capítulos y para evitar la excesiva longitud de ciertos apartados, algunas definiciones se han movido al Anexo ``Definiciones y acornimos'' y aparecen en el texto de la siguiente manera -> [D\ref{def1}].
