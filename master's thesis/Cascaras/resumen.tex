\chapter*{Resumen}

La computación en la nube, la Inteligencia artificial y el tratamiento de grandes volúmenes de datos se revelaron como tecnologías profundamente disruptivas en el panorama tecnológico de los últimos años. En este trabajo se estudió el estado de la cuestión de estas tecnologías en profundidad, analizando su utilidad intrínseca y explorando las distintas metodologías, técnicas y servicios existentes, aportando también una implementación práctica para examinar su capacidad de aportar un beneficio público tangible.

Se examinó la disponibilidad de conjuntos de datos públicos a nivel europeo como base para el proyecto, cumpliendo con los nuevos marcos legales que este impone, A partir de ello, se elaboró una metodología replicable que describió el recorrido de los datos desde su origen hasta la generación de valor público, haciendo uso de la computación en la nube y las capacidades avanzadas de los algoritmos de Aprendizaje automático, así como las tecnologias de Auto-Machine Learning que las grandes nubes publicas proporcionan. Esto permitió identificar las configuraciones más eficientes y óptimas para la construcción de estas soluciones, comparando con el resto de opciones disponibles. De igual manera, permitió desarrollar un enfoque general del estado de las tres tecnologías estudiadas en la actualidad. %[TODO] 195 palabras, revisar y usar modelos principalmente utilizados, o algo mas especifico

Los ficheros de GitHub se encuentran en el siguiente repositorio:  \url{https://github.com/crismo04/TFM-cloud-soliutions-to-public-data/}
	
\section*{Palabras clave}
   
\noindent Big Data, Computación en la nube, Inteligencia Artificial, Machine Learning, Open Data, Tratamiento de datos, Valor público
   


