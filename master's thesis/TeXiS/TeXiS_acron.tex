%---------------------------------------------------------------------
%
%                        TeXiS_acron.tex
%
%---------------------------------------------------------------------
%
% TeXiS_acron.tex
% Copyright 2009 Marco Antonio Gomez-Martin, Pedro Pablo Gomez-Martin
%
% This file belongs to TeXiS, a LaTeX template for writting
% Thesis and other documents. The complete last TeXiS package can
% be obtained from http://gaia.fdi.ucm.es/projects/texis/
%
% This work may be distributed and/or modified under the
% conditions of the LaTeX Project Public License, either version 1.3
% of this license or (at your option) any later version.
% The latest version of this license is in
%   http://www.latex-project.org/lppl.txt
% and version 1.3 or later is part of all distributions of LaTeX
% version 2005/12/01 or later.
%
% This work has the LPPL maintenance status `maintained'.
% 
% The Current Maintainers of this work are Marco Antonio Gomez-Martin
% and Pedro Pablo Gomez-Martin
%
%---------------------------------------------------------------------
%
% Contiene  los  comandos  para  generar  el listado de acrónimos
% documento.
%
%---------------------------------------------------------------------
%
% NOTA IMPORTANTE:  para que la  generación de acrónimos  funcione, al
% menos  debe  existir  un  acrónimo   en  el  documento.  Si  no,  la
% compilación  del   fichero  LaTeX  falla  con   un  error  "extraño"
% (indicando  que  quizá  falte  un \item).   Consulta  el  comentario
% referente al paquete glosstex en TeXiS_pream.tex.
%
%---------------------------------------------------------------------


% Redefinimos a español  el título de la lista  de acrónimos (Babel no
% lo hace por nosotros esta vez)

\def\listacronymname{Lista de acrónimos}

% Para el glosario:
% \def\glosarryname{Glosario}


\newacronym{gcp}{GCP}{Google Cloud Platform}
\newacronym{aws}{AWS}{Amazon Web Services}
\newacronym{azure}{Azure}{Microsoft Azure}
\newacronym{oci}{OCI}{Oracle Cloud Infrastructure}
\newacronym{ibm}{IBM}{International Business Machines}
\newacronym{cpu}{CPU}{Unidad Central de Procesamiento}
\newacronym{gpu}{GPU}{Unidad de Procesamiento Gráfico}
\newacronym{tpu}{TPU}{Unidad de Procesamiento Tensorial}
\newacronym{ram}{RAM}{Memoria de Acceso Aleatorio}
\newacronym{ssd}{SSD}{Disco de Estado Sólido}
\newacronym{cdn}{CDN}{Red de Distribución de Contenidos}
\newacronym{ddos}{DDoS}{Ataque de Denegación de Servicio Distribuido}
\newacronym{dns}{DNS}{Sistema de Nombres de Dominio}
\newacronym{ssl}{SSL}{Capa de Conexión Segura}
\newacronym{sase}{SASE}{Acceso Seguro al Borde del Servicio}
\newacronym{kv}{KV}{Key-Value}
\newacronym{r2}{R2}{R2 Storage}
\newacronym{d1}{D1}{D1 Database}
\newacronym{turn}{TURN}{Traversal Using Relays around NAT}
\newacronym{mlops}{MLOps}{Operaciones de Aprendizaje Automático}
\newacronym{ai}{IA}{Inteligencia Artificial}
\newacronym{ml}{ML}{Aprendizaje Automático}
\newacronym{nlp}{NLP}{Procesamiento del Lenguaje Natural}
\newacronym{api}{API}{Interfaz de Programación de Aplicaciones}
\newacronym{cnmc}{CNMC}{Comisión Nacional de los Mercados y la Competencia}
\newacronym{oecd}{OECD}{Organización para la Cooperación y el Desarrollo Económicos}
\newacronym{rgpd}{RGPD}{Reglamento General de Protección de Datos}
\newacronym{lgd}{LGD}{Ley de Gobernanza de Datos}
\newacronym{hipaa}{HIPAA}{Ley de Portabilidad y Responsabilidad de Seguros de Salud}
\newacronym{ccpa}{CCPA}{Ley de Privacidad del Consumidor de California}
\newacronym{ogda}{OGDA}{Ley de Datos Abiertos del Gobierno}
\newacronym{pipl}{PIPL}{Ley de Protección de Información Personal}
\newacronym{dsl}{DSL}{Ley de Seguridad de Datos}
\newacronym{ine}{INE}{Instituto Nacional de Estadística}
\newacronym{csv}{CSV}{Valores Separados por Comas}
\newacronym{json}{JSON}{Notación de Objetos de JavaScript}
\newacronym{eu}{UE}{Unión Europea}
\newacronym{eeuu}{EEUU}{Estados Unidos}


% Si se  va a generar  la tabla de  contenidos (el índice  habitual) y
% también vamos a  generar la lista de acrónimos  (ambas decisiones se
% toman en  función de  la definición  o no de  un par  de constantes,
% puedes consultar config.tex  para más información), entonces metemos
% en la  tabla de contenidos una  entrada para marcar  la página donde
% está el índice de palabras.

\ifx\generatoc\undefined
\else
   \addcontentsline{toc}{chapter}{\listacronymname}
\fi


% Generamos la lista de acrónimos (en realidad el índice asociado a la
% lista "acr" de GlossTeX)

\printglosstex(acr)

% Variable local para emacs, para  que encuentre el fichero maestro de
% compilación y funcionen mejor algunas teclas rápidas de AucTeX

%%%
%%% Local Variables:
%%% mode: latex
%%% TeX-master: "../Tesis.tex"
%%% End:
