%---------------------------------------------------------------------
%
%                          config.tex
%
%---------------------------------------------------------------------
%
% Contiene la  definición de constantes  que determinan el modo  en el
% que se compilará el documento.
%
%---------------------------------------------------------------------
%
% En concreto, podemos  indicar si queremos "modo release",  en el que
% no  aparecerán  los  comentarios  (creados  mediante  \com{Texto}  o
% \comp{Texto}) ni los "por  hacer" (creados mediante \todo{Texto}), y
% sí aparecerán los índices. El modo "debug" (o mejor dicho en modo no
% "release" muestra los índices  (construirlos lleva tiempo y son poco
% útiles  salvo  para   la  versión  final),  pero  sí   el  resto  de
% anotaciones.
%
% Si se compila con LaTeX (no  con pdflatex) en modo Debug, también se
% muestran en una esquina de cada página las entradas (en el índice de
% palabras) que referencian  a dicha página (consulta TeXiS_pream.tex,
% en la parte referente a show).
%
% El soporte para  el índice de palabras en  TeXiS es embrionario, por
% lo  que no  asumas que  esto funcionará  correctamente.  Consulta la
% documentación al respecto en TeXiS_pream.tex.
%
%
% También  aquí configuramos  si queremos  o  no que  se incluyan  los
% acrónimos  en el  documento final  en la  versión release.  Para eso
% define (o no) la constante \acronimosEnRelease.
%
% Utilizando \compilaCapitulo{nombre}  podemos también especificar qué
% capítulo(s) queremos que se compilen. Si no se pone nada, se compila
% el documento  completo.  Si se pone, por  ejemplo, 01Introduccion se
% compilará únicamente el fichero Capitulos/01Introduccion.tex
%
% Para compilar varios  capítulos, se separan sus nombres  con comas y
% no se ponen espacios de separación.
%
% En realidad  la macro \compilaCapitulo  está definida en  el fichero
% principal tesis.tex.
%
%---------------------------------------------------------------------


% Comentar la línea si no se compila en modo release.
% TeXiS hará el resto.
% ¡¡¡Si cambias esto, haz un make clean antes de recompilar!!!
\def\release{1}


% Descomentar la linea si se quieren incluir los
% acrónimos en modo release (en modo debug
% no se incluirán nunca).
% ¡¡¡Si cambias esto, haz un make clean antes de recompilar!!!
\def\acronimosEnRelease{1}
% \def\generaacronimos

% Descomentar la línea para establecer el capítulo que queremos
% compilar

% \compilaCapitulo{01Introduccion}
% \compilaCapitulo{02EstructuraYGeneracion}
% \compilaCapitulo{03Edicion}
% \compilaCapitulo{04Imagenes}
% \compilaCapitulo{05Bibliografia}
% \compilaCapitulo{06Makefile}

% \compilaApendice{01AsiSeHizo}

% Variable local para emacs, para  que encuentre el fichero maestro de
% compilación y funcionen mejor algunas teclas rápidas de AucTeX
%%%
%%% Local Variables:
%%% mode: latex
%%% TeX-master: "./Tesis.tex"
%%% End:
